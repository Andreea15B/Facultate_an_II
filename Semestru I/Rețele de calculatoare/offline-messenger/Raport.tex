\documentclass[runningheads]{llncs}
\usepackage[left=30mm,top=30mm,right=30mm,bottom=30mm]{geometry}
\usepackage{graphicx}
\usepackage{datetime2}
\usepackage{listings, lstautogobble}
\usepackage{xcolor}
\definecolor{light-gray}{gray}{0.95}
\lstnewenvironment{queryl}[1][] 
 {\lstset{autogobble=true, backgroundcolor=\color{light-gray}, language=C++, frame=shadowbox, escapechar=`, #1}}
 {}
\renewcommand\refname{Bibliografie}


\title{Offline Messenger}
\author{Bucătaru Andreea A2}
\institute{Facultatea de Informatică, Iași}

\setlength{\parindent}{4em}

\begin{document}

\maketitle
\noindent
\makebox[\linewidth]{\small 14 Decembrie 2018}

\section{Introducere}
\paragraph{}
Offline Messenger este o aplicație care permite schimbul de mesaje între utilizatori care sunt conectați și oferă
funcționalitatea trimiterii mesajelor și către utilizatorii offline, acestora din urmă apărându-le mesajele atunci
când se vor conecta la server. De asemenea, utilizatorii au posibilitatea de a trimite un răspuns (reply) în mod
specific la anumite mesaje primite. Aplicația oferă și istoricul conversațiilor pentru și cu fiecare utilizator
în parte.

\section{Tehnologii utilizate}
\paragraph{}
Aplicația trebuie să țină evidența userilor conectați. Având în vedere acest lucru, vom folosi protocolul de comunicare TCP, deoarece
acesta ne asigură realizarea unei conexiuni între client și server. TCP asigură ordinea și retransmiterea pachetelor în caz de pierdere, oferind siguranță.
\vspace{-4mm}
\paragraph{}
Ca limbaj de programare, aplicația este scrisă în C/C++.

\section{Arhitectura aplicației}
\paragraph{}
Am folosit principiile programării orientate pe obiect pentru o mai bună organizare și modificare a 
funcționalităților aplicației.
\vspace{-4mm}
\paragraph{}
Aplicația este formată din două părți: server și client. 
\vspace{-4mm}
\paragraph{}
Serverul va gestiona clienții în mod concurent. De asemenea, acesta se va ocupa de
transmiterea datelor preluate de la un client la un altul în două moduri: online sau offline. 
\vspace{-4mm}
\paragraph{}
Partea de client se va ocupa doar de preluarea inputului de la utilizator și trimiterea acestuia către server pentru a fi procesat.
Pe lângă acest lucru, va furniza utilizatorului informații primite de la server.

\section{Detalii de implementare}
\paragraph{}
Modelul TCP este implementat concurent, folosindu-se $socket$-uri pentru comunicarea între procese.
\vspace{-4mm}
\paragraph{}
În server, pentru a gestiona clienții în mod concurent, vom crea un proces separat, cu $fork()$, menit să se ocupe de fiecare client:

\begin{queryl}[linewidth=16.2cm, frame=single]
    void Server::listenClients() {
        listen(serverDescriptor, 5);
        while (1) {
          clientDescriptor = accept(serverDescriptor, &from, &length);
          handleClient(clientDescriptor); // servim in mod concurent clientii
        }
    }
\end{queryl}

\begin{queryl}[linewidth=16.2cm, frame=single]
    void Server::handleClient(int clientDescriptor) {
        if (fork() != 0) return; // parintele continua listenClients
        while (1) {
          myRead(clientDescriptor, user.state);
          if(user.state == 1) { // log in
            myRead (clientDescriptor, user.username);
            myRead (clientDescriptor, user.password);
            int found = checkUsernamePassword(user.username, user.password);
            myWrite(clientDescriptor, found);
          }
          else if(user.state == 2) { // sign up
            myRead (clientDescriptor, user.username);
            myRead (clientDescriptor, user.password);
            int found = checkUsername(user.username);
            myWrite(clientDescriptor, found);
            if(!found)
              addNewUser(user.username, user.password); // il adaug in fisier
          }
          else if(user.state == 3) break; // quit
          else if(user.state == 5) { // chat
            userToChat = chooseUserToChat(clientDescriptor, user);
            chat(clientDescriptor, user, userToChat);
          }
          else if(user.state == 6) to be continued... // view history
          else if(user.state == 7) break; // log out
        }
      }
\end{queryl}

\paragraph{}
Partea de client se va ocupa de citirea comezilor și procesarea acestora:

\begin{queryl}[linewidth=12cm, frame=single]
void Client::listenInput() {
    while(!quited) {
        showMenu();
        cin >> state;
        validateInput(input);
        processInput(input);
    }
}
\end{queryl}

\begin{queryl}[linewidth=12cm, frame=single]  
void Client::processInput(string input) {
    switch (state) {
        case 1: login(); break;
        case 2: signUp(); break;
        case 3: quit(); break;
        case 4: chat(); break;
        case 5: viewHistory(); break;
        case 6: logOut(); break;
    }
}
\end{queryl}

\section{Concluzii}
\paragraph{}
$fork()$ este un mecanism costisitor din punctul de vedere al timpului, de aceea soluția propusă ar putea fi îmbunătățită folosind 
$thread$-uri.
\vspace{-4mm}
\paragraph{}
Aplicația ar fi mult mai ușor de utilizat dacă partea de client ar avea și o interfață grafică.
\vspace{-4mm}
\paragraph{}
Pentru stocarea datelor utilizatorilor și pentru ușurința decodificării acestora, am putea salva informațiile într-o bază de date sau în fișiere de tip $xml$.

\bibliographystyle{splncs}
\bibliography{mybibliography}

\begin{thebibliography}{}

\bibitem{}
Cursurile de la Rețele de calculatoare: 
https://profs.info.uaic.ro/$\sim$computernetworks/cursullaboratorul.php

\bibitem{}
Laboratoarele de la Rețele de calculatoare:
https://profs.info.uaic.ro/$\sim$eonica/rc/

\bibitem{}
https://stackoverflow.com/

\end{thebibliography}
\end{document}
